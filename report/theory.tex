\chapter{Theory}

\section{Strongly Correlated Systems}
\subsection{Exercise 1}
\paragraph*{(a)} Let $a_i^\dagger$ be the creation operator on orbital $i$ and $a_i$ the annhilation operator. The canonical anticommutation relations are
\begin{align}
    \{a_i,\, a_j\} &= \{a_i^\dagger,\, a_j^\dagger\} = 0 \\
    \{a_i,\, a_j^\dagger\} &= \delta_{i,\, j} \mathds{1}.
\end{align}

\paragraph*{(b)} One transformation is
\begin{align}
    c_{i,0} &= a_i + a_i^\dagger \\
    c_{i,1} &= \mathrm{i}(a_i - a_i^\dagger),
\end{align}
where $i$ is the orbital index.

\paragraph*{(c)} Majorana fermions satisfy the following anticommutation relation
\begin{equation}
    \{c_{i,\alpha}c_{i,\beta}\} = \delta_{i,j}\delta_{\alpha,\beta} \mathds{1}.
\end{equation}


\subsection{Exercise 2}
\paragraph*{(a)} Using the Jordan-Wigner transformation, a fermionic operator $a_j$ or $a_j^\dagger$ becomes a $j$-local qubit operator, since it acts non trivially on $j$ sites.

\paragraph*{(b)} Thanks to the fact of storing only partial sums of qubits occupation, a fermionic operator translates into a $O(\log(j))$-local qubit operator.